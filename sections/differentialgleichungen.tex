\renewcommand{\arraystretch}{1}
\section{Differentialgleichungen\formelbuch{552}}

\subsection{Lösen von Differentialgleichungen 1.Ordnung}

\subsubsection{Picard-Lindelöf}
Die Funktion $f(x, u, u_1, ..., u_{n-1})$ sei in einer Umgebung der Stelle $(x_0, y_0, y_1, ..., y_{n-1}) \in \mathbf{R^{n+1}}$ stetig und besitzt dort stetige partielle Ableitungen
nach $u, u_1, ..., u_{n-1}$ dann existiert in einer geeigneten Umgebung des Anfangspunktes $x_0$ genau eine Lösung des Anfangswertproblems\\
$y^{(n)} = f(x, y, y', ...,y^{(n-1)})$ mit $y(x_0) = y_0, y'(x_0) = y_1, ..., y^{(n-1)}(x_0) = y_{n-1}$ \\ \\
$\frac{\partial f}{\partial y}$ ... $\frac{\partial f}{\partial f^{(n-1)}}$ endlich beschränkt $\Rightarrow$ eindeutige Lösbarkeit


\subsubsection{Trennung von Variabeln / Separation \formelbuch{554}}
\begin{tabular}{p{4cm}p{1.5cm}p{10.5cm}}
\textbf{Form:} $y' = f(x) g(y)$ &
\textbf{Vorgehen:}              &
$\frac{y'}{g(y)} = f(x)$, nun ist die DGL beidseitig nach x integrierbar\\  &&
($dy = y'(x) dx$): $\int \frac{1}{g(y)} dy = \int f(x) dx$ 
\end{tabular}

\subsubsection{Lineartermsubstitution/separierte Lösung\formelbuch{554}}
\begin{tabular}{p{4cm}p{1.5cm}p{10.5cm}}
\textbf{Form:} $y'=f(ax+by+c)$   &
\textbf{Vorgehen:}               &
1. Substitution: $z=ax+by+c \qquad z'=a+by' =a+bf(z)$\\ &&
$\int\limits_{x_0}^{x}\frac{z'}{a+bf(z)}d\tilde{x} = \int 1 d\tilde{x} \Rightarrow \int\limits_{z_0}^{z}\frac{1}{a+bf(\tilde{z})}d\tilde{z} = \int\limits_{x_0}^{x}1 d\tilde{x} \qquad [d\tilde{z} = \underbrace{(a+by')}_{z'} d\tilde{x}]$
\end{tabular}

\subsubsection{Gleichgradigkeit\formelbuch{554}}
\begin{tabular}{p{4cm}p{1.5cm}p{15cm}}
\textbf{Form:} $y'=f(\frac{y}{x})$ &
\textbf{Vorgehen:}                &
1. Substitution: $z=\frac{y}{x} \qquad
z'=\frac{1}{x}(f(z)-z) \qquad
y'=f(z) \qquad
dz=y'(x)dx$ 
\end{tabular}

\subsubsection{Lineare Differentialgleichungen 1. Ordnung \formelbuch{555}}
\begin{tabular}{p{4.5cm}p{1.5cm}p{10.5cm}}
\textbf{Form:} $ y'+f(x)y = \underbrace{g(x)}_{\text{Störglied}} $ &
\textbf{Vorgehen:}                 &
$ y=e^{-\int f(x) dx}(k+\int g(x)e^{\int f(x)dx}dx) \qquad (k\in\mathbf{R})$
\end{tabular}

\subsection{Lineare Differentialgleichung 2. Ordnung mit konstanten Koeffizienten \formelbuch{573}}
\begin{tabular}{p{8cm}p{8cm}}
\textbf{Form:} $y''+a_1\cdot y'+a_0\cdot y=f(x)$  &
\textbf{Störglied:} $f(x)$\\
\textbf{Homogene Differentialgleichung:} $f(x)=0$ &
\textbf{Inhomogene Differentialgleichung:} $f(x)\neq 0$
\end{tabular}

\subsubsection{Allgemeine Lösung einer homogenen DGL:\quad\subsubadd{$\quad Y_H$}}
\textbf{Charakteristisches Polynom}
$\qquad\underline{\lambda^2+a_1\cdot\lambda+a_0=0}$ \hspace{1cm}von
$\qquad\underline{y''+a_1\cdot y'+a_0\cdot y=0}$ 
$\qquad(\lambda_{1,2} = -\frac{a_1}{2} \pm \frac{\sqrt{a_1^2 - 4a_0}}{2})$\\ \\

\begin{tabular}{p{2cm}p{5cm}p{6cm}p{4cm}}
$(D > 0)$ &
Falls $\lambda_1\neq \lambda_2$ und $\lambda_{1,2} \in R$: &
$Y_H=Ae^{\lambda_1x}+Be^{\lambda_2x}$ & 
$\rbrace$ starke Dämpfung\\

$(D = 0)$ &
Falls $\lambda_1=\lambda_2$ und $\lambda_{1,2} \in R$: &
$Y_H=e^{\lambda_1x}(A+B\cdot x)$ & 
$\rbrace$ aperiodischer Grenzfall\\

$(D < 0)$ &
Falls $\lambda_{1,2}=-\frac{a_1}{2}\pm j\alpha$: &
$Y_H=e^{-\frac{1}{2}a_1x}(Acos(\alpha x) +Bsin(\alpha x))$ &
$\rbrace$ schwache Dämpfung / Schwingfall \\
\end{tabular}

\begin{tabular}{p{2cm}p{5cm}p{2cm}p{4cm}}
	Eigenfrequenz: & $\omega = \alpha = \frac{\sqrt{|a_1^2 - 4a_0|}}{2}$ &
	Dämpfung: &  $|\delta| = |\lambda|$\\
\end{tabular}

\subsubsection{Allgemeine Lösung einer inhomogenen DGL:\quad\subsubadd{$y=Y_H+y_P$}}

\subsubsection{Grundlöseverfahren einer inhomogenen DGL:\quad\subsubadd{$\quad y_P$}}
Homogene DGL: $g(x) = Y_H$ mit den Anfangsbedingungen $g(x_0) = 0; g'(x_0) = 1$. Wenn möglich $x_0 = 0$.\\
$$y_P(x)=\int\limits_{x_o}^{x} g(x+x_0-t)\cdot f(t)dt$$	

\subsubsection{Vorgehen bei einer inh. DGL mit Störgliedtabelle: }
	Alle Schritte werden anhand diesem Beispiel erklärt: $\textcolor{red}{y'' + 3y' + 2y = 3e^{-2x}}$
	\begin{compactenum}
		\item 	$Y_H$ mit $\lambda_1$ und $\lambda_2$ berechnen \\
				\textcolor{red}{$y'' + 3y' + 2y = 0 \Rightarrow \lambda^2 + 3\lambda + 2 = 0 \Rightarrow \lambda_1 = -1$ und $\lambda_2 = -2 \Rightarrow Y_H = Ae^{-x} + Be^{-2x} \Rightarrow Y_{H1} = Ae^{-x}$ und $ Y_{H2} = Be^{-2c}$}
		\item 	Anhand der Störglied Tabelle $y_p$ bestimmen \\
				$\textcolor{red}{y_p = Ae^{-2x}}$
		\item 	Testen ob $y_p(x) = Y_{H1}$ oder $y_p(x) = Y_{H2}$ ist. 
				Falls Bedingung(en) zutreffen: $y_p(x)= y_p(x) * x^{\text{Anzahl zutreffende Bedingungen}}$ \\
				\textcolor{red}{$ y_p(x) = Y_{H2} \Rightarrow y_p(x) = Ae^{-2x}x $}
		\item 	$y_p$ ableiten und in die DGL einsetzen \\
				\textcolor{red}{$y_p'=(-2Ax + A)e^{-2x}$ und $y_p''=(4Ax - 4A)e^{-2x} \Rightarrow (4Ax - 4A)e^{-2x} + 3(-2Ax + A)e^{-2x} + 2Axe^{-2x} = 3e^{-2x}$}
		\item Gleichung kürzen und nach x-Potenzen ordnen \\
				\textcolor{red}{$(4A - 6A + 2A)xe^{-2x} + (3A - 4A)e^{-2x}=3e^{-2x}$}
		\item 	Koeffizienten bestimmen: \\				
				\begin{tabular}{ll}
					\textcolor{red}{$(3A - 4A) = 3$} & \textcolor{red}{$(3A - 4A)$} kommt 3 mal in $g(x)$ vor \\
					\textcolor{red}{$(4A - 6A + 2A) = 0$} & \textcolor{red}{$(4A - 6A + 2A)$} kommt 0 mal in $g(x)$ vor \\
				\end{tabular} \\
				$\textcolor{red}{A = -3}$
		\item 	Koeffizienten in $y_p$ einsetzen
				$\textcolor{red}{y_p = -3e^{-2x}}$
		\item 	Wenn das Störglied $f(x)$ aus mehreren Teilen besteht (z.B. $x^2e^x + x$), Störglied auseinander nehmen und in zwei Teile $x^2e^x$ und $x$ unterteilen und Schritt 3 - 6 wiederholen
		\item 	$y = Y_H + y_{p1} + y_{p2} + \dots$
	\end{compactenum}
	
\paragraph{Störgliedtabelle}
	\begin{tabular}{|p{8cm}|p{10cm}|}
		\hline 	
			Störglied $g(x)$ & Ansatz $y_p$ \\
		\hline
			$k$ (Konstante) & $A$ \\
		\hline
			$x^n$ & \multirow{2}{*}{$A_n*x^n + \dots + A_1*x + A_0$} \\
			$p_n(x) = b_n*x^n + \dots + b_1*x + b_0$ & \\
		\hline
			$k*e^{m*x}$ & $A*e^{m*x}$ \\
		\hline	
			$k*cos(b*x)$ & \multirow{3}{*}{$A*cos(b*x) + B*sin(b*x)$} \\
			$k*sin(b*x)$ & \\
			$k_1*cos(b*x) + k_2*sin(b*x)$ & \\
		\hline
			$k*e^{m*x}*cos(b*x)$ & \multirow{3}{*}{$e^{m*x}*(A*cos(b*x) + B*sin(b*x))$} \\
			$k*e^{m*x}*sin(b*x)$ & \\
			$e^{m*x}*(k_1*cos(b*x) + k_2*sin(b*x)$ & \\
		\hline
			$k*cosh(b*x)$ & \multirow{3}{*}{$A*cosh(b*x) + B*sinh(b*x)$} \\
			$k*sinh(b*x)$ & \\
			$k_1*cosh(b*x) + k_2*sinh(b*x)$ & \\
		\hline
			$k*e^{m*x}*cosh(b*x)$ & \multirow{3}{*}{$e^{m*x}*(A*cohs(b*x) + B*sinh(b*x))$} \\
			$k*e^{m*x}*sinh(b*x)$ & \\
			$e^{m*x}*(k_1*cosh(b*x) + k_2*sinh(b*x)$ & \\
		\hline
			$k*x*e^{mx}$ & $(A*x+B)*e^{m*x}$ \\
		\hline
			$p_n(x)*e^{m*x}$ & $(A_n*x^n + \dots + A_1*x + A_0)*e^{mx}$ \\
		\hline
			$x*(k_1*cos(b*x) + k_2*sin(b*x))$ & $(A_1*x+B_1)*cos(b*x) + (A_2*x+B_2)*sin(b*x)$ \\
		\hline
			$x*e^{mx}*(k_1*cos(b*x) + k_2*sin(b*x))$ & $e^{mx}*((A_1*x+B_1)*cos(b*x) + (A_2*x+B_2)*sin(b*x))$ \\
		\hline
			$x*(k_1*cosh(b*x) + k_2*sinh(b*x))$ & $(A_1*x+B_1)*cosh(b*x) + (A_2*x+B_2)*sinh(b*x)$ \\
		\hline
			$x*e^{mx}*(k_1*cosh(b*x) + k_2*sinh(b*x))$ & $e^{mx}*((A_1*x+B_1)*cosh(b*x) + (A_2*x+B_2)*sinh(b*x))$ \\
		\hline
	\end{tabular}
	\newpage

%%%%%%%%%%%%%%%%%%%%%%%%%%%%%%%%%%%%%%%%%%%%%%%%%%%%%%%%%%%%%%%%%%%%%%%%%%%%%%%%%%%%%%%%%%%%%%%%
%%%%%%%%%%%%%%%%%%%%%%%%%%%%%%%%%%%%%%%%%%%%%%%%%%%%%%%%%%%%%%%%%%%%%%%%%%%%%%%%%%%%%%%%%%%%%%%%

%\newpage

\subsubsection{Superpositionsprinzip}
$f(x)=c_1f_1(x)+c_2f_2(x)$\\
\begin{tabular}{p{8cm}p{8cm}}
$y_1$ ist spezielle Lösung der DGL &
$y_1''+a_1\cdot y_1'+a_0\cdot y_1=c_1f_1(x)$ \\
$y_2$ ist spezielle Lösung der DGL &
$y_2''+a_1\cdot y_2'+a_0\cdot y_2=c_2f_2(x)$ \\
dann ist:                          &
$y_P=c_1y_1+c_2y_2$\\
\end{tabular}

\subsection{Lineare Differentialgleichung n. Ordnung mit konstanten Koeffizienten \formelbuch{554}}
	\begin{tabular}{p{1.5cm}p{8cm}}
		\textbf{Form:} &
		$\sum\limits_{k=0}^na_ky^{(k)}= y^{(n)}+a_{n-1}\cdot y^{(n-1)}+\ldots +a_0\cdot y=f(x)$\\
	\end{tabular}

\subsubsection{n-verschiedene Homogene Lösungen}
	\begin{tabular}{lll}
		Fall a: r reelle Lösungen $\lambda_1$: 
			& $y_1=e^{\lambda_1x}$, $y_2=xe^{\lambda_1x}$, \ldots
			,$y_r=x^{r-1}e^{\lambda_1x}$ 
			& Starke Dämpfung / Kriechfall\\
		Fall b: $k$ komplexe Lösungen $\lambda_2=\alpha +j\beta$: 
			&$y_1=e^{\alpha x}\cos(\beta x)$, \ldots, $y_k=e^{\alpha x}x^{k-1}\cos(\beta
		x)$
			& Schwache Dämpfung /\\
			&$y_{k+1}=e^{\alpha x}\sin(\beta x)$, \ldots, $y_{2k}=e^{\alpha
		x}x^{k-1}\sin(\beta x)$
			& Schwingfall\\
	\end{tabular}
	$Y_H = Ay_1 + By_2 + Cy_3 + ... + Ny_n$

\subsubsection{Allgemeinste Lösung des partikulären Teils:}
	$$\underbrace{\sum_{k=0}^n a_k y^{(k)}}_{f(y,y',y'',\ldots)} = \underbrace{e^{\alpha x} (p_{m1}(x) \cos (\beta x) + q_{m2}(x) \sin (\beta x))}_{\text{Störglied}} \qquad \lambda \text{ aus Homogenlösung}$$
	Unterscheide die Lösungen des charakteristischen Polynoms ($\lambda$):\hspace{5.5cm}mit m = max(m1, m2)\\
	\begin{tabular}{p{8cm}p{8.5cm}}
		Fall a: $\alpha + j\beta \neq \lambda$, so ist &
		$y_P = e^{\alpha x}(r_m(x)\cos(\beta x) + s_m(x) \sin(\beta x))$\\
		Fall b: $\alpha + j\beta$  ist u-fache Lösung von $\lambda$, so ist &
		$y_P = e^{\alpha x} x^u (r_m(x) \cos(\beta x) + s_m(x) \sin(\beta x))$\\
		&
		u-fache Resonanz
	\end{tabular}

\subsubsection{Grundlöseverfahren}
	\begin{tabular}{p{12cm}p{5cm}}
		$\begin{pmatrix}
		g(x_0)=  & 0 & = & Ay_1(x_0)+By_2(x_0)+\ldots +Ny_n(x_0)\\
		g'(x_0)= & 0 & = & Ay_1'(x_0)+By_2'(x_0)+\ldots +Ny_n'(x_0)\\
		\vdots  & \vdots & \\                            
		g^{(n-1)}(x_0)= & 1 & = & Ay_1^{(n-1)}(x_0)+By_2^{(n-1)}(x_0)+\ldots
		+Ny_n^{(n-1)}(x_0)
		\end{pmatrix}$ &
		\begin{minipage}[t]{5cm}
			ergibt $c_1,\ldots ,c_n$ für\\
			$y_{P}(x)=\int\limits_{x_0}^x{g(x+x_0-t)f(t)dt}$
		\end{minipage}
	\end{tabular}

\subsubsection{Anfangswertproblem}
	$y(x_0) = y_0 \qquad y'(x_0) = y_1 \qquad y''(x_0) = y_2 \qquad \dots \qquad y^{(n-1)}(x_0) = y_{n-1}$

\subsection{Lineare Differentialgleichungssysteme erster Ordnung mit konstanten Koeffizienten}
	\begin{tabular}{p{8cm}p{8cm}}
		\textbf{Form:}& $	\begin{matrix} \dot{x}=ax+by+f(t) \\ \dot{y}=cx+dy+g(t) \end{matrix} = \left(\begin{matrix} \dot{x} \\ \dot{y} \end{matrix}\right) = 
					\left(\begin{matrix} a & b \\ c & d \end{matrix}\right) \left(\begin{matrix} x \\ y \end{matrix}\right) + \left(\begin{matrix} f(t) \\ g(t) \end{matrix}\right)$ \\
	
	
		\textbf{Die allgem. Lösung ergibt sich aus der DGL:}&
		$\underbrace{\ddot{x}-(a+d)\dot{x}+(ad-bc)x=\dot{f}(t)-df(t)+bg(t)}_{\text{normale DGL 2.Ordnung} \rightarrow \text{nach $x$ auflösen}}$\\
		& $y=\frac{1}{b}(\dot{x}-ax-f(t)))$\\
	
		\textbf{Anfangsbedinung:} &
		$x_0(t_0) = x_0, \dot{x}_0(t_0) = ax_0 + by_0 + f(t_0)$
	\end{tabular} \\ \\
	\textbf{Anordnung beachten!} Gesuchte Grösse immer zu oberst (in diesem Fall ist die gesuchte Grösse $x$)

\subsection{Faltung \formelbuch{802}}
	$f(x) = \int\limits_0^x f_1(x-t)f_2(t) dt \qquad$ Schreibweise $f = f_1 *  f_2$

